% !TEX root = ../master.tex
\chapter{Fundamentals}
\label{chap:sota}

TODO forcast on chapter

\section{Computer Vision}

TODO

\ref{fig:sota:imageengineering}
\begin{figure}[hbt]
	\centering
	\includegraphics[width=0.8\textwidth, keepaspectratio]{08_Chapter01_fig1-4}
	\caption{\label{fig:sota:imageengineering} Levels of image engineering. 
	Reprinted from \textcite[][Chapter~1]{zhang2017imageprocessing}}
\end{figure}

\subsection{Object Detection}

TODO
edge detection, outline detection
foreground background mask from video image
background substraction

\subsection{Volume Estimation}

TODO
single view, multiview, depth sensors
estimation by pixel area
interpolation of depth information by combinding views
bounding / section box
full 3d reconstruction
\ref{fig:sota:mulitviewtop}

\begin{figure}[hbt]
	\centering
	\includegraphics[width=1.0\textwidth, keepaspectratio]{resources/multiview}
	\caption{\label{fig:sota:mulitviewtop}Concept of depth reconstruction in a 2D multi-view situation.
	Based on and adapted from \textcite[][]{sonaten2011volume}}
\end{figure}


\section{Elevator Control}
TODO shor introtuction, definition and what are important aspects about it

\subsection{System Components}
TODO
- shaft and movement mechanism
- motors
- car / cabin / convoyance
- doors
- in cabing panel
- panel on each floor
- advanced cabin sensory
- advanced floor sensory
- control electronic per elevator
-- motor control (cabin dispatch)
-- door control
-- security system
- grou controller
TODO find picture or reference

\autocite[][]{strobl1999controller}

\subsection{System Classifications}
TODO
- by amount of elevators
- by type of sensory to haul the car
TODO how many elevators parallel
TODO additional travel information: how many people inside and outside, up / down / floor select inside / outside, everyone presses?, video? 
destination control system
- type of goods / people transported \autocite[][p.~141]{unger2015aufzuege}

\subsection{Passenger Traffic Patterns}
Passenger traffic patterns are typical streams of passengers going from one floor to another.
Those patterns are reoccurring regularly based on the time of the day.
The movement of passengers can be described with three directions: \emph{incoming}, \emph{outgoing}, and \emph{inter-floor} traffic.
With those directions the traffic patterns can be described, as they focus on one of the directions, but also incorporates components of passengers heading in other directions \autocite[][p.~259]{siikonen1993simulation}.
According to multiple sources, three general patterns are present in office buildings \autocite[][pp.~1--2]{beers2015arrivals}
\autocite[][pp.~6--7]{axelsson2013strategies}
\autocite[][p.~194]{unger2015aufzuege}
\autocite[][p.~14]{siikonen1997models}:

\begin{itemize}
    \item \textbf{Up-peak traffic} In the morning the majority of passengers in an office building constitutes to an incoming traffic.
    They arrive at the entrance lobby and travel to different upper floors to fill the building.
    The lift cars potentially need to stop at every level and return to the lobby without passengers to pick up new ones. 
    This reduces the utilized capacity and puts a high load on the elevator system to convoy all passengers in time.
    \item \textbf{Down-peak traffic} In the evening the majority of the passengers are leaving the office building and travel from arbitrary upper floors to the main lobby.
    Most of the passengers constitute to outgoing traffic.
    The elevator system needs to pick up passengers at every level. 
    Reverse to the up-peak traffic, the cars are empty when returning from the lobby.
    Similar to the up-peak traffic this situation induces a high stress on the system.
    \item \textbf{Inter-floor traffic} All other traffic that is neither incoming nor outgoing can be considered inter-floor traffic. Passengers that travel from one floor to another but are not entering or leaving the building are the majority in this situation.
\end{itemize}

%Figure \ref{fig:sota:trafficpatterns} depicts the former mentioned traffic components.
Figure \ref{fig:sota:traffictimes} shows an example of how the traffic components contribute to overall traffic in an office building. A clear peak of incoming traffic in the morning and outgoing traffic in the evening are visible.
This information can be used to model simulations and predict traffic in various circumstances.
It has significant impact on the scheduling principles employed for each situation in order to improve the efficiency of an elevator system.

%\begin{figure}[hbt]
%	\centering
%	\includegraphics[width=0.8\textwidth, keepaspectratio]{resources/trafficpatterns}
%	\caption[]{\label{fig:sota:trafficpatterns} Typical categories of passenger traffic: Incoming, Outgoing and Inter-floor.
%	Reprinted from \textcite[][p.~259]{siikonen1993simulation}}
%\end{figure}

\begin{figure}[hbt]
	\centering
	\includegraphics[width=0.8\textwidth, keepaspectratio]{resources/traffictimes}
	\caption{\label{fig:sota:traffictimes} Exemplary traffic component profile for an office building.
	Reprinted from \textcite[][p.~14]{siikonen1997models}}
\end{figure}

TODO layout the figures
TODO or use \textcite[][p.~12]{sorsa2005destination} as broader image 

\subsection{Control Strategies}
TODO

\autocite[][pp.~3--4,10]{beers2015arrivals}
- stopping policies / strategies
- parking polcies
- call allocation

\autocite[][pp.~3--6]{axelsson2013strategies}
- collective control
- zoneing
- search based
- rule based
- genetic algorithm




\subsection{Performance Criteria}

TODO

TODO
metrics from passenger perspective
\autocite[][p.~10]{beers2015arrivals}
\autocite[][p.~7]{hakonen2003simulation}
\autocite[][pp.8-9]{siikonen1997models}




\begin{itemize}
    \item \textbf{Passenger waiting time} TODO - passenger waiting time
    \item \textbf{Passenger ride time} TODO - passenger travel (riding) time
    \item \textbf{Total journey time} TODO  - total service / journey time / time to destination
\end{itemize}

TODO
general metrics
\autocite[][p.~194]{unger2015aufzuege}
\autocite[][p.~7]{hakonen2003simulation}
\autocite[][pp.8-9]{siikonen1997models}

\begin{itemize}
    \item \textbf{Handling capacity} TODO - handling capacity
    \item \textbf{Round trip time} TODO - round trip time
    \item \textbf{Interval time} TODO - interval
    \item \textbf{Hall call time} TODO - call time / hall call time / call waiting time
    \item \textbf{5 minute interval} TODO - 5min interval
    \item \textbf{Total stops} TODO - total stops
\end{itemize}

TODO
economic metrics



\begin{itemize}
    \item \textbf{Energy consumption} TODO - energy consumption
    \item \textbf{Capacity utilization} TODO - capacity utilization in space and weight 
        (usualy one cabin is only used 60-80\% could be lower if objects in it)
        \autocite[][p.~194]{unger2015aufzuege}
        \autocite[][p.~7]{hakonen2003simulation}
        influenced by empty rides
\end{itemize}

TODO
for each maxima and average are important metrics
also used for traffic modeling and finding 


\subsection{Simulational Optimization Approaches}
TODO
model building etc: here is a lot of research in the topic utilizing theoretical multivariable model optimization, simulational approaches and heuristical approaches in real world (refs?)

\autocite[][pp.~7--11]{beers2015arrivals}
\autocite[][p.~193]{unger2015aufzuege}


TODO
