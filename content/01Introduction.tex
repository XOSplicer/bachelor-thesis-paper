% !TEX root = ../master.tex
\chapter{Introduction}
\label{chap:intro}

\section{Computer Vision and Elevators}

TODO
- general intro to why camera vision is gaining importance in public places for survialance, statistics, monitoring live reactrion. 
- iot mevement with cameras and many other sensors
- and why elevator movement is an topic of economic interest.
- there is a lot of research in the topic utilizing theoretical multivariable model optimization, simulational approaches and heuristical approaches in real world ["copy" to sota and ref papers]

- history of elevators dates way back
- rapid development since the industrial revolutions
- now object of everyday interest and importance for high rise builduings where they are elemental
- current research into optimization to make them more efficient from the prespective of waiting time and economics
- maybe leave out section heading





\section{Context}

- andere paper referenzieren vorheriger research?
- DXC, digital service innovation team, Startup autobahn
- startup nicht mehr da, kunde abgesprungen??


TODO

\section{Motivation}


TODO

where does the problem come from?
why is this work relevant?
volume detectoin might hold the opportunity to further improve efficiency with the degree of usage on elevators
limited by 80\% oder auch 60-70\% rule ??, unused capacity
also detection of bigger items such as janitors cars might be useful


mögliche kunden Otis, Kone, Thyssen, Schindler)

\section{Goals and Boundaries}

TODO Problem Statement (maybe own section)

TODO scope and non-scope

TODO research questions

possibly not in scope: detect emergencies if person is laiing down, maybe not


\section{Structure of this Document}

TODO

\begin{itemize}
    \item Chapter \ref{chap:sota} describes the current state of the art in the fields of elevator control and movement optimization, as well as the basics of image processing that are needed for the conduct of this project.
    \item Chapter \ref{chap:design} 
    \item Chapter \ref{chap:impl}
    \item Chapter \ref{chap:concl} gives a short summary of the conduced method and the outcomes of the project. The findings are discussed and the execution of the project is reflected upon. Finally an outlook into future developments in the field and the opportunities and follow-up actions for DXC Technology and the Digital Service Innovation Team are outlined.
\end{itemize}