% !TEX root = ../master.tex
\chapter{Introduction}
\label{chap:intro}

%\section{Computer Vision and Elevators}

The usage of elevator constructions to lift goods and people vertically dates back a long history and gained significant importance since the industrialization.
Nowadays modern high-rise buildings would not be sustainable without elevation systems appropriate to handle large amount of business people on their way to and from the office. 
The issue of optimizing the throughput of passengers and reducing the their waiting time is of everyday economic interest since it is critical for the business people using it to use their time most efficiently. 
Therefore the topic of movement optimization for elevators is under active academic research with direct impact for manufacturers.
Such optimizations of the control mechanism depend on many parameters including timings of the system, the traffic patterns of the passengers and the kind of input mechanisms to detect passengers and terminal destination.

In this thesis a possibility to enhance this control mechanism by utilizing computer--based camera vision is proposed.
Currently the general trend of computer vision is emerging in many fields with real--world applications
and techniques of varying complexities are employed to support a broad range of use cases. 
In industrial systems camera data is used to control and inspect production processes. 
When monitoring solutions for public or private places are considered such use cases include among others surveillance, tracking,
smart home, and statistical applications.
Especially in \enquote{smart} environments video data is combined with other sensory data to gain insight into behavioral patterns and trigger actions in that environment.

Here the possibilities are explored to which extend
computer vision can be used to improve the movement control of elevators by estimating the free spacial volume in the elevator cabins and how much space the passengers waiting for the car would take up.
This could allow for a better utilization of the cabins and an improved passenger--to--car mapping with reduced waiting times for passengers.

\section{Context}

This bachelor thesis is conducted in the \emph{Digital Service Innovation} team at \emph{DXC Technology}.
The team is focused helping clients to develop novel, digitized business models.
The team is actively involved in the \emph{Startup Autobahn} program organized by \emph{Plug and Play}.
The program provides a business platform to connect technology start--ups and global industry cooperations 
by running a 100 day cycle in which demo projects can be conducted to test out how 

- DXC, digital service innovation team, Startup autobahn
- startup nicht mehr da, kunde abgesprungen??


- andere paper referenzieren vorheriger research?
TODO

\section{Motivation}


TODO


where does the problem come from?
why is this work relevant?
auch wenn der kunde weg ist, will DXC weiter das thema verfolgen und damit zu möglicher weiteren kunden gehen und vllt dazu startups ins boot holen
volume detectoin might hold the opportunity to further improve efficiency with the degree of usage on elevators
limited by 80\% oder auch 60-70\% rule ??, unused capacity
also detection of bigger items such as janitors cars might be useful


mögliche kunden Otis, Kone, Thyssen, Schindler (nennbar?)

\section{Goals and Boundaries}

TODO Problem Statement (maybe own section)

TODO scope and non-scope

TODO research questions

possibly not in scope: detect emergencies if person is laiing down, maybe not


\section{Structure of this Document}

TODO

\begin{itemize}
    \item Chapter \ref{chap:sota} describes the current state of the art in the fields of elevator control and movement optimization, as well as the basics of image processing that are needed for the conduct of this project.
    \item Chapter \ref{chap:design} 
    \item Chapter \ref{chap:impl}
    \item Chapter \ref{chap:concl} gives a short summary of the conduced method and the outcomes of the project. The findings are discussed and the execution of the project is reflected upon. Finally an outlook into future developments in the field and the opportunities and follow-up actions for DXC Technology and the Digital Service Innovation Team are outlined.
\end{itemize}