% !TEX root = ../master.tex
\chapter{Introduction}
\label{chap:intro}

\section{Computer Vision and Elevators}

%\begin{figure}
%\begin{equation}
%f(x)=x^2
%\label{eqn:e2}
%\end{equation}
%\caption{Quadratic Function}
%\end{figure}

% \lipsum[1-10]
\autocite{reddy2013foreground}

TODO
- general intro to why camera vision is gaining importance in public places for survialance, statistics, monitoring live reactrion. 
- iot mevement with cameras and many other sensors
- and why elevator movement is an topic of economic interest.
- there is a lot of research in the topic utilizing theoretical multivariable model optimization, simulational approaches and heuristical approaches in real world ["copy" to sota and ref papers]


\section{Context}

- andere paper referenzieren vorheriger research?
- DXC Startup autobahn 
- startup abgesprungen

TODO

\section{Motivation}


TODO

where does the problem come from?
why is this work relevant?
volume detectoin might hold the opportunity to further improve efficiency with the degree of usage on elevators
limited by 80\% oder auch 60-70\% rule ??, unused capacity
also detection of bigger items such as janitors cars might be useful

possibly in scope: detect emergencies if person is laiing down, maybe not

mögliche kunden Otis, Kone, Thyssen, Schindler)

\section{Goals and Boundaries}

TODO Problem Statement (maybe own section)

TODO scope and non-scope

TODO research questions

\section{Structure of this Document}

TODO

\begin{itemize}
    \item Chapter \ref{chap:sota}
    \item Chapter \ref{chap:design}
    \item Chapter \ref{chap:impl}
    \item Chapter \ref{chap:concl}
\end{itemize}