% !TEX root = ../master.tex
\chapter{Introduction}
\label{chap:intro}

The usage of elevator constructions to lift goods and people dates back to a long history and gained significant importance since the industrialization.
Nowadays modern high-rise buildings would not be sustainable without elevation systems capable to handle large amounts of business people on their way to and from the office. 
The issue of optimizing the throughput of passengers and reducing their waiting time is of everyday economic interest since it is critical for the business people using it to use their time most efficiently. 
Therefore movement optimization for elevators is under active academic research with direct impact for manufacturers.
Such optimizations of the control mechanism depend on many parameters including timings of the system, the traffic patterns of the passengers and the kind of input mechanisms to detect passengers and terminal destination.

In this thesis, a possibility to enhance this control mechanism by utilizing computer-based camera vision is proposed.
Currently, the general trend of computer vision is emerging in many fields with real-world applications
and techniques of varying complexities are employed to support a broad range of use cases. 
In industrial systems, camera data is used to control and inspect production processes. 
When monitoring solutions for public or private places are considered such use cases include among others surveillance, tracking,
smart home, and statistical applications.
Especially in \enquote{smart} environments, video data is combined with other sensory data to gain insight into behavioral patterns and trigger actions in that environment.

Here the possibilities are explored to which extend
computer vision can be used to improve the movement control of elevators by estimating the free spacial volume in the elevator cabins and how much space the passengers waiting for the car would take up.
This could allow better utilization of the cabins and an improved passenger-to-car mapping with reduced waiting times for passengers.

\section{Context and Motivation}
\label{sec:into:context}

This bachelor thesis is conducted in the \emph{Digital Service Innovation} team at \emph{DXC Technology}.
The team focuses on helping clients to develop novel, digitized business models.
The team is actively involved in the \emph{Startup Autobahn} program organized by \emph{Plug and Play}.
The program provides a business platform to connect technology start-ups and global industry corporations
by running a 100-day cycle in which demo projects can be conducted to test out how the start-ups can work with the companies and how their technologies can be used.

The idea for this thesis evolved in this context as DXC got interested in a start-up that deals with computer vision and found a possible client in the elevator industry.
However, this first engagement did not lead to a joint project.
But still, DXC is keen to follow the idea and find new partners and clients to collaborate with.
Therefore this project is nevertheless relevant for the team even though there is no immediate financial benefit associated with it, 
but rather it opens up opportunities for new partnerships and client engagements when suitable technology partners are found.
According to \textcite[][p.~4]{unger2015aufzuege} the four \enquote{biggest} companies in the elevator business are \emph{Otis, Kone, Schindler and Thyssen} which therefore are potential clients to benefit from the ideas proposed in this thesis.

For academia, this research is interesting because elevator systems typically only utilize
60-70\% of their theoretical capacity in one ride \autocite[][p.~194]{unger2015aufzuege}. 
When looking at the whole-day average, this figure might even be lower considering empty ride.
Using the additional passenger information generated by a vision system capable of detecting the volume of passengers and other cargo might hold the opportunity to improve this efficiency further.

\section{Goals and Boundaries}
\label{sec:intro:goals}

In order to set a clear focus of this thesis, concrete questions that shall be answered are helpful.
Those provide guidance in the process of determining the methods employed in later investigation steps.
In general two main research questions are identified:

\begin{itemize}
    \item Which computer vision techniques can be used to detect persons and objects within an elevator car and in front of the lift and estimate their spacial volume?
    \item How can the information acquired by a system that uses such methods constitute to the optimization of algorithms that control the movement plan for elevators?
\end{itemize}

To contribute to this general questions, it is useful to also evaluate some secondary questions:

\begin{itemize}
    \item The volume of what kind of other objects are except for humans are interesting for this topic?
    \item Under which circumstances is it useful to integrate such a system into an elevator system?
    \item To which extent can the usage lead to improvements in utilization, waiting time, ride time and economic effort?
    \item Which of the named optimization goals are targetable by this kind of analysis and are they desired from a passenger perspective?
\end{itemize}

In order to answer those questions, an explicit plan to follow is laid out in chapter \ref{chap:design} which is executed and implemented in chapter \ref{chap:impl}. 
This will also determine which analysis, implementation and evaluation actions are taken. 
In general, an experimental approach is used, where an exemplary implementation of the volume detection system is performed which is tested on real-life sample footage. 
An analytical approach is used to derive the possibilities to integrate the volume data into the scheduling of an elevator.

Additionally, to keeping the project size manageable, it is necessary to exclude aspects from the  investigation explicitly. 
Especially, it is of no interest to track and identify individual persons and analyze their behavior. 
Even though some kind of surveillance could be established this way, it is not desired for this thesis.
Furthermore, considerations about the market potential for the solution and how it can be targeted to customers are performed. 
Also, no long-time analysis of passenger traffic is performed based on the volume measurements.
This includes that no machine learning techniques are applied.

\section{Structure of this Document}

This thesis is structured into multiple chapters that first build a general understanding of the problem and the technologies needed to solve it and leads to a final implementation.

\begin{itemize}
    \item This chapter gives an introduction into the topic and outlines the problem statement which is worked on in the thesis.
    \item Chapter \ref{chap:sota} describes the current state of the art in the fields of elevator control and movement optimization, as well as the basics of image processing that are needed for the conduct of this project. It lays out which methods, technologies, and tools can be used to approach the problems in this project.
    \item Chapter \ref{chap:design} explains the methods used to conduct research for the project and plans a concept for the solution. An artifact centered approach is used, which plans to build and evaluate two products: one to perform the visual volume measurement and one to show the effectiveness of integrating this data into the scheduling of an elevator.
    \item Chapter \ref{chap:impl} describes how the proposed solutions are implemented and tested. The solution is reflected upon from a functional and economic perspective. The details of the implementation planned in the chapter beforehand are laid out and the artifacts are tested according to the set criteria.
    \item Chapter \ref{chap:concl} gives a short summary of the conducted method and the outcomes of the project. The findings are discussed and the execution of the project is reflected upon. Finally, an outlook into future developments in the field and the opportunities and follow-up actions for DXC Technology and the Digital Service Innovation Team are outlined.
\end{itemize}