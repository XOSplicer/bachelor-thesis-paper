% !TEX root = ../master.tex
\chapter{Conclusions}
\label{chap:concl}

This chapter summarizes the outcomes of the thesis at hand 
and reflects on the conducted methods.
Key ideas of the previous chapters are pointed out and a discussion regarding the solution quality and alternative solutions is held.
Lastly a suggestion for the next steps at DXC Technology regarding the utilization of this thesis is given.

\section{Summary}

This thesis explored the possibilities to support the movement control of elevator systems with the help of a camera system to estimate the volume occupied in the cabins.
First, the theoretical background in \ac{2D} and \ac{3D} computer vision have been explained.
After that, two major artifacts have been designed, created and evaluated by following the principles of design science research.
Primarily a design for a volume recognition system inside elevator cabin has been conducted.
The system uses techniques which include foreground-masking by background subtraction with the \ac{MOG} method and most importantly the technique of voxel based volume intersection.
The system was partially implemented and tested on a real elevator cabin to show the general feasibility of the chosen approach.

As a secondary artifact a new elevator control strategy has been presented.
It extends the common collective control strategy by giving priority to delivering cargo objects in order to free up the cabin quickly.
A simulation which models an elevator system with discrete events has been used to evaluate the performance of this control strategy.
It has been show that in a single-elevator set-up with constant arrival rates of passengers and cargo objects the new strategy delivers 35.52~\% more cargo objects than collective control.
It has a 18.40~\% lower average ride time which comes at the cost of a 30.56~\% higher waiting time.
Possible use cases include buildings where the elevator commonly conveys passengers and cargo objects.
Examples are hospitals, auxiliary lifts and fabrication plants.

\section{Discussion and Critical Reflection}

TODO Discussion regarding outcomes
TODO Discussion regarding external Results - fragezeichen
TODO discussion regarding initial questions
 - all are implicitly answered
 - which computer vision techniques: background subtraction,volume intersection
 - optimization of movement: priority to cargo objects
 - other interesting objects: carts, forklifts hospital beds
 - useful circumstances: single elevator shared by caro and passenegrs
 - extent of usefulness shown by simulation
 - optimization goal: amount of delivered cargo objects amongst others
TODO discussion regarding alternative solutions
 
TODO Discussion of simulation parameter, strategy and program flow correctness, discussion of simulation output correctness
TODO  Discussion regarding internal Execution and chosen methods - einhaltung von design science research

\section{Further Work}

The presented visual solution is far from production-ready and only showed the general feasibility of the approach.
If the project of this thesis is continued, a complete implementation of the visual system is desirable, which uses a correct perspective projection and realizes the proposed blob detection and classification.
However, based on the shown results, DXC Technology can try to get possible customers from the elevator business interested in developing innovative ideas together with them.
They can explore many more use cases of camera systems in elevators and team up with computer vision professionals or start-ups to put a solution into place, which meets the customers needs.
