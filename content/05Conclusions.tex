% !TEX root = ../master.tex
\chapter{Conclusions}
\label{chap:concl}

This chapter summarizes the outcomes of the thesis at hand 
and reflects on the conducted methods.
Key ideas of the previous chapters are pointed out and a discussion regarding the solution quality and alternative solutions is held.
Lastly a suggestion for the next steps at DXC Technology regarding the utilization of this thesis is given.

\section{Summary}

This thesis explored the possibilities to support the movement control of elevator systems with the help of a camera system to estimate the volume occupied in the cabins.
First, the theoretical background in \ac{2D} and \ac{3D} computer vision have been explained.
After that, two major artifacts have been designed, created and evaluated by following the principles of design science research.
Primarily a design for a volume recognition system inside elevator cabin has been conducted.
The system uses techniques which include foreground-masking by background subtraction with the \ac{MOG} method and most importantly the technique of voxel based volume intersection.
The system was partially implemented and tested on a real elevator cabin to show the general feasibility of the chosen approach.

As a secondary artifact a new elevator control strategy has been presented.
It extends the common collective control strategy by giving priority to delivering cargo objects in order to free up the cabin quickly.
A simulation which models an elevator system with discrete events has been used to evaluate the performance of this control strategy.
It has been show that in a single-elevator set-up with constant arrival rates of passengers and cargo objects the new strategy delivers 35.52~\% more cargo objects than collective control.
It has a 18.40~\% lower average ride time which comes at the cost of a 30.56~\% higher waiting time.
Possible use cases include buildings where the elevator commonly conveys passengers and cargo objects.
Examples are hospitals, auxiliary lifts and fabrication plants.

\section{Discussion and Critical Reflection}

In the introduction of this thesis (section \ref{sec:intro:goals}) several research questions have been stated,
which over the over the course of this document have been implicitly answered.
It has been explained, that the computer vision techniques of background subtraction and volume intersection are one possibility to detect passengers and other objects inside a elevator cabin. 
The movement of that elevator can then optimized by altering its control strategy based on whether cargo objects are present and prioritize their delivery.
Other interesting objects that might be detected have been deduced from specific use cases and include janitor carts, forklifts or hospital beds.
It has been found out, 
that it  is useful to integrate the proposed visual system into an elevator in buildings where passengers and cargo objects share a single lift.
The extent of the usefulness of the integration has been shown by running a simulation. 
It turns out that a desirable optimization goal is the fast delivery of cargo objects.

In the consideration for both artifacts only one particular solution each has been explored in greater detail.
However, many more use cases and approaches would have been interesting.
The decision to focus on this particular use cases has been based on based on the authors interest and a more thorough decision process enhance the scientific demands of this thesis.
Alternative solutions for the visual system could be found in depth sensor technology, feature based depth reconstruction with stereo vision.
However these solutions would come at a higher expense or complexity.
In the initial problem exploration, a installation of cameras in front of the cabin on each floor has also been proposed, but the idea was not pursued further. 
Alternative solutions for the conducted elevator simulation could be found by investing in industrial software instead of modeling the elevator system by manually. 

The conducted simulation was kept simple with specific building parameters 
and models the elevator system by using discrete events, which yields a presumable correct implementation.
Moreover, the simulation only models one particular situation and therefore a lot of premises went into the parameters of the system.
Some parameters were not even considered in the implementation, even though they were specified.
It would also have been possible to model the physical behavior in more detail of the system and take more parameters into account.
Also, only a simple traffic pattern and basic control strategies have been implemented. 
A more detailed modeling in this respects could lead to more informed results. 
Taking more metrics out of the simulation can also create a deeper understanding of the properties of the simulation, like the direct measurement of capacity utilization or the total journey time.
The results generated by the simulation do not contradict the authors intuition and lie in an expected range. 
However the results give only a vague indication of how the proposed scheduling algorithm would perform in reality.

The chosen methodology of design science research proved capable to 
conduct the research in this thesis.
It provides a broad guideline to the project execution 
and ensures the produced artifacts are not only designed and implemented, but also evaluated and presented.
In particular also the re-circulation of the gained insight into the body of knowledge of academia is demanded.
However, the results of this thesis are not likely to be published in scientific conferences or journals.

\section{Further Work}

The presented visual solution is far from production-ready and only showed the general feasibility of the approach.
If the project of this thesis is continued, a complete implementation of the visual system is desirable, which uses a correct perspective projection and realizes the proposed blob detection and classification.
However, based on the shown results, DXC Technology can try to get possible customers from the elevator business interested in developing innovative ideas together with them.
They can explore many more use cases of camera systems in elevators and team up with computer vision professionals or start-ups to put a solution into place, which meets the customers needs.
