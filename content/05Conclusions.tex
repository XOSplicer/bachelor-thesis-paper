% !TEX root = ../master.tex
\chapter{Conclusions}
\label{chap:concl}

This chapter summarizes the outcomes of the thesis at hand 
and reflects on the conducted methods.
Key ideas of the previous chapters are pointed out and a discussion regarding the solution quality and alternative solutions is held.
Lastly a suggestion for the next steps at DXC Technology regarding the utilization of this thesis is given.

\section{Summary}

This thesis explored the possibilities to support the movement control of elevator systems with the help of a camera system to estimate the volume occupied in the cabins.
First, the theoretical background in \ac{2D} and \ac{3D} computer vision have been explained.
After that, two major artifacts have been designed, created and evaluated by following the principles of design science research.
Primarily a design for a volume recognition system inside elevator cabin has been conducted.
The system uses techniques which include foreground-masking by background subtraction with the \ac{MOG} method and most importantly the technique of voxel based volume intersection.
The system was partially implemented and tested on a real elevator cabin to show the general feasibility of the chosen approach.

As a secondary artifact a new elevator control strategy has been presented.
It extends the common collective control strategy by giving priority to delivering cargo objects in order to free up the cabin quickly.
A simulation which models an elevator system with discrete events has been used to evaluate the performance of this control strategy.
It has been show that in a single-elevator set-up with constant arrival rates of passengers and cargo objects the new strategy delivers 35.52~\% more cargo objects than collective control.
It has a 18.40~\% lower average ride time which comes at the cost of a 30.56~\% higher waiting time.
Possible use cases include buildings where the elevator commonly conveys passengers and cargo objects.
Examples are hospitals, auxiliary lifts and fabrication plants.

\section{Discussion and Critical Reflection}

TODO Discussion regarding outcomes
- only inside lift, not infront
TODO Discussion regarding external Results - fragezeichen

In the introduction of this thesis several research questions have been stated,
which over the over the course of this document have been implicitly answered.
It has been explained, that the computer vision techniques of background subtraction and volume intersection are one possibility to detect passengers and other objects inside a elevator cabin. The movement of that elevator can then optimized by altering its control strategy based on whether cargo objects are present and prioritize their delivery.
Other interesting objects that might be detected have been deduced from specific use cases and include janitor carts, 

TODO discussion regarding initial questions
 - all are implicitly answered
 - which computer vision techniques: background subtraction,volume intersection
 - optimization of movement: priority to cargo objects
 - other interesting objects: carts, forklifts hospital beds
 - useful circumstances: single elevator shared by caro and passenegrs
 - extent of usefulness shown by simulation
 - optimization goal: amount of delivered cargo objects amongst others
TODO discussion regarding alternative solutions
 - only one particular use case with one particular solution was explored further, however many more use cases and approaches would have been possible, the decision to follow this particular one were not perfectly argumented (das sollte schöner formuliert werden), alternative solutions for the visual system could be found in depth sensors, or feature based scenen reconstruction with multi stereo vision. but they come hat either heigher cost or higher complexety
TODO Discussion of simulation parameter, strategy and program flow correctness, discussion of simulation output correctness
- simululation was kept simple with a specific building parameters, but still modeled a elevator system, presumably correctly, more detailed modeling would be possibly in modeling advanced traffic pattern, and maybe testing also directional collective strategy
- more metrics could have been taken, like capacity utilization in passengers and kilograms, tital jourey time
- the results are in an expected frame relative to each other
TODO  Discussion regarding internal Execution and chosen methods 
- design science is a very capable method to be used for this research
- einhaltung von design science research - rückführung des wissens in academia begrenzt
- performance measures for visual system not quantitative
- result presentation not infront of relevant external business or academia, ich bin werder verkäufer, noch gehe ich zu einer konference

\section{Further Work}

The presented visual solution is far from production-ready and only showed the general feasibility of the approach.
If the project of this thesis is continued, a complete implementation of the visual system is desirable, which uses a correct perspective projection and realizes the proposed blob detection and classification.
However, based on the shown results, DXC Technology can try to get possible customers from the elevator business interested in developing innovative ideas together with them.
They can explore many more use cases of camera systems in elevators and team up with computer vision professionals or start-ups to put a solution into place, which meets the customers needs.
