% !TEX root = ../master.tex
\chapter{Conclusions}
\label{chap:concl}

This chapter summarizes the outcomes of the thesis at hand 
and reflects on the conducted methods.
Key ideas of the previous chapters are pointed out and a discussion regarding the solution quality and alternative solutions is held.
Lastly a suggestion for the next steps at DXC Technology regarding the utilization of this thesis is given.

\section{Summary}

TODO
we saw:
- theoretical background of 2d and 3d computer vision as well as elevator systems
- design for an visual system to be used in an elevator
- partial implementation and testing of said system
 - used techniques include foreground masking by background subtraction with the \ac{MOG} method and morphological post-processing and most importantly the technique of voxel based colume intersection


- the thoughts that go into simulating an elevator system
- a new scheduling algorithm suited for cargo lifts used by passengers and cargo
which prioritizes cargo, 
- the performance evaluation of this algorithm in the simulation
-according to own simulation it delivers 35.52~\% more cargo than collective control, and has a 18.40~\% lower average wait time but comes at the cost of a 30.56~\% higher waiting time.
- possible use cases include hospitals, auxiliary lifts and fabrication plants

\section{Discussion and Critical Reflection}

TODO Discussion regarding outcomes
TODO Discussion regarding external Results
discussion regarding initial questions
discussion regarding alternative solutions
Discussion of simulation parameter, strategy and programflow correctness, discussion of simulation output corecctness

TODO  Discussion regarding internal Execution and chosen methods


\section{Further Work}

TODO
this project: possible to do mathematical correct and complete implementation
but how needs this is far from production ready
dxc: try to get customers interested and develop new ideas with them based on this work
team up with computer vision experts if necessary

