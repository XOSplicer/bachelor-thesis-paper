% !TEX root = ../master.tex

\chapter{Simulation Source Code (excerpts)}
\label{app:sim}

\begin{lstlisting}[caption={Main function of the simulation program}, label={lst:app:simmain}]
fn main() {
    let b = BuildingParameters::default();
    let l = LiftParameters::default();
    let s = SimulationParameters::default();
    println!("{:#?}\n{:#?}\n{:#?}\ntotal_ticks: {}\n", &b, &l, &s, s.total_ticks());

    let results =
        rayon::iter::repeatn((), s.simulations) // enable parallel execution
        .enumerate()
        .inspect(|(num, _)| println!("running simulation {}", num))
        .map(|_| TrafficGenerator::new(&s, &b).arrivals())
        .map(|t| (
            run_single_simulation::<SequentialControlStrategy>(&b, &l, &s, &t),
            run_single_simulation::<CollectiveControlStrategy>(&b, &l, &s, &t),
            run_single_simulation::<AdaptedControlStrategy>(&b, &l, &s, &t),
            ))
        .collect::<Vec<_>>();
    let invalid = results.iter()
        .filter(|&(a, b, c)| a.is_err() || b.is_err() || c.is_err())
        .count();
    let valid: Vec<_> = results.into_iter()
        .filter(|&(ref a, ref b, ref c)| a.is_ok() && b.is_ok() && c.is_ok())
        .map(|(a, b, c)| (a.unwrap(), b.unwrap(), c.unwrap()))
        .collect();
    let valid_a = valid.iter().map(|&(ref a, ref _b, ref _c)| a);
    let valid_b = valid.iter().map(|&(ref _a, ref b, ref _c)| b);
    let valid_c = valid.iter().map(|&(ref _a, ref _b, ref c)| c);

    println!("invalid results: {}", invalid);
    println!("{}", SequentialControlStrategy::name());
    print_stats(valid_a);
    println!("{}", CollectiveControlStrategy::name());
    print_stats(valid_b);
    println!("{}", AdaptedControlStrategy::name());
    print_stats(valid_c);

    println!("\ndone.");
}
\end{lstlisting}


\begin{lstlisting}[caption={Implementation of a single simulation run}, label={lst:app:simsinglerun}]
fn run_single_simulation<S: ControlStrategy>(
    building: &BuildingParameters,
    lift: &LiftParameters,
    sim: &SimulationParameters,
    traffic: &Vec<Vec<TrafficItem>>,
) -> Result<SimulationResults, String> {

    println!("running simulation: {:?}", S::name());
    let mut system = ElevatorSystem::new(building, lift, sim);
    let mut state = S::State::default();
    let mut results = SimulationResults::default();

    results.passengers_in_traffic = traffic.iter()
        .flat_map(|t| t)
        .filter(|&t| t.kind == TrafficItemKind::Passenger)
        .count();
    results.cargo_in_traffic = traffic.iter()
        .flat_map(|t| t)
        .filter(|&t| t.kind == TrafficItemKind::Cargo)
        .count();

    while system.ticks < sim.total_ticks() {
        let last_ticks = system.ticks;
        let (next_state, action) = S::action(state, &system);
        state = next_state;
        match action {
            ControlAction::MoveUp => {
                let time = system.move_up()?;
                system.ticks += sim.time_to_ticks(time);
            },
            ControlAction::MoveDown => {
                let time = system.move_down()?;
                system.ticks += sim.time_to_ticks(time);
            },
            ControlAction::OpenDoors => {
                let time = system.open_doors()?;
                results.stops += 1;
                system.ticks += sim.time_to_ticks(time);
            },
            ControlAction::CloseDoors => {
                let time = system.close_doors()?;
                system.ticks += sim.time_to_ticks(time);
            },
            ControlAction::LoadTrafficItems(amount) => {
                let mut to_load = Vec::with_capacity(amount);
                for _ in 0..amount {
                    if let Some(i) = system.floor_queues[system.lift_floor]
                        .pop_front()
                    {
                        to_load.push(i);
                    }
                }
                let (time, rest) = system.load_traffic_items(to_load)?;
                for i in rest {
                    system.floor_queues[system.lift_floor]
                        .push_front(i);
                }
                system.ticks += sim.time_to_ticks(time);
            },
            ControlAction::UnloadTrafficItems => {
                let current_floor = system.lift_floor;
                let (time, items) = system.unload_traffic_items_for_floor(current_floor)?;
                results.passengers_delivered += items.iter()
                    .filter(|&i| i.kind == TrafficItemKind::Passenger)
                    .count();
                results.cargo_delivered += items.iter()
                    .filter(|&i| i.kind == TrafficItemKind::Cargo)
                    .count();
                system.ticks += sim.time_to_ticks(time);
            },
            ControlAction::Nothing => {
                system.ticks += 1;
            }
        }

        let next_tick = cmp::min(system.ticks, sim.total_ticks());
        let tick_diff = next_tick - last_ticks;

        // update waiting times
        for id in system.floor_queues.iter()
            .flat_map(|v| v.iter())
            .map(|i| i.id)
        {
            *results.waiting_ticks.entry(id).or_insert(0) += tick_diff;
        }

        // update ride times
        for id in system.lift_traffic_items.values()
            .flat_map(|v| v.iter())
            .map(|i| i.id)
        {
            *results.ride_ticks.entry(id).or_insert(0) += tick_diff;
        }

        // advance ticks and update floor queues with traffic items
        let new_traffic_items = traffic[last_ticks..next_tick]
            .iter()
            .flat_map(|v| v)
            .cloned();
        for i in new_traffic_items {
            system.floor_queues[i.from_floor].push_back(i)
        }

    }

    // remove ride times of passenegers that did not reach destination yet
    // and are in the lift
    for id in system.lift_traffic_items.values()
        .flat_map(|v| v.iter())
        .map(|i| i.id)
    {
        results.ride_ticks.remove(&id);
    }

    // remove waiting times of passenegrs still waiting
    for id in system.floor_queues.iter()
        .flat_map(|v| v.iter())
        .map(|i| i.id)
    {
        results.waiting_ticks.remove(&id);
    }

    Ok(results)
}
\end{lstlisting}

\chapter{Volume Intersection Source Code (excerpts)}

\begin{lstlisting}[caption={Main function of the volume intersection program}, label={lst:app:volmain}]
def main():
    print_versions()

    cv2.namedWindow('sliders', flags=cv2.WINDOW_NORMAL)
    cv2.createTrackbar('open_kernel','sliders',3,101,nop)
    cv2.createTrackbar('dilate_kernel','sliders',17,101,nop)

    view_3d = init_3d_plot()
    create_3d_room(view_3d)
    create_3d_cams(view_3d)
    vol_3d = create_3d_volume(view_3d)
    side_3d = create_3d_side(view_3d)

    caps = [ cv2.VideoCapture(c['file']) for c in CAPTURES ]
    # bgss = [ cv2.createBackgroundSubtractorMOG2() for _ in caps ]
    bgss_pp = [ cv2.createBackgroundSubtractorMOG2() for _ in caps ]

    video_out_res = (1920, 480)
    if OUTPUT:
        fourcc = cv2.VideoWriter_fourcc(*'XVID')
        video_out = cv2.VideoWriter('output.avi',fourcc, 25.0, video_out_res)

    while True:
        # read all capture images at this moment
        raws = [ c.read()[1] for c in caps]

        # convert to grayscale
        gry = [ cv2.cvtColor(c, cv2.COLOR_BGR2GRAY) for c in raws ]
        gry_preview = multi_preview(gry)
        #cv2.imshow('gry_preview', gry_preview)

        # get parameters from trackbars
        k_size = cv2.getTrackbarPos('open_kernel','sliders')
        k_size = k_size - k_size % 2 +1
        d_size = cv2.getTrackbarPos('dilate_kernel','sliders')
        d_size = d_size - d_size % 2 +1

        # pre process images and appy background subtractor
        pp = [ cv2.GaussianBlur(r, (k_size,k_size), 0) for r in raws ]
        bgs_pp_masks = [ bgss_pp[i].apply(pp[i]) for i in range(len(raws)) ]

        # post process foreground masks
        open_kernel = np.ones((k_size,k_size),np.uint8)
        dilate_kernel = np.ones((d_size,d_size),np.uint8)
        bgs_pp_masks_t = [ cv2.threshold(b, 254, 255, cv2.THRESH_BINARY)[1] for b in bgs_pp_masks]
        bgs_pp_masks_b = [ cv2.dilate(cv2.morphologyEx(b, cv2.MORPH_OPEN, open_kernel), dilate_kernel, iterations = 10)\
             for b in bgs_pp_masks_t ]
        bgs_pp_preview = multi_preview(bgs_pp_masks_b)
        #cv2.imshow('bgs_pp_preview', bgs_pp_preview)

        # draw masks A B C to the side, othorgonal (semi correct)
        scaled_masks = draw_3d_side(side_3d, bgs_pp_masks_b)

        # create orthorgonal  intersection of A B C
        vol_poss = np.array([ \
            [ \
                u * ROOM_SCALE_MPVX['width'],\
                v * ROOM_SCALE_MPVX['depth'],\
                w * ROOM_SCALE_MPVX['height']\
            ] \
            for u in range(ROOM_DIM_VX['width']) \
            for v in range(ROOM_DIM_VX['depth']) \
            for w in range(ROOM_DIM_VX['height']) \
            if  scaled_masks[0][w][v] == 255 \
            and scaled_masks[1][w][u] == 255 \
            and scaled_masks[2][w][v] == 255 ])
        vol_3d.setData(pos=vol_poss)

        # create combined preview
        image_3d = convertQImageToMat(view_3d.readQImage())
        scale = 480 / image_3d.shape[0]
        image_3d_preview = cv2.resize(image_3d, (0,0), fx=scale, fy=scale)
        image_3d_preview = cv2.cvtColor(image_3d_preview, cv2.COLOR_BGRA2BGR)
        full_preview = np.hstack((\
            cv2.cvtColor(np.vstack((gry_preview, bgs_pp_preview)), cv2.COLOR_GRAY2BGR), \
            image_3d_preview \
            ))
        cv2.imshow('full_preview', full_preview)
        out = np.zeros((video_out_res[1], video_out_res[0], 3), dtype=np.uint8)
        out[:, 0:full_preview.shape[1], :] = full_preview
        if OUTPUT:
            video_out.write(out)


        # unnecessary distance measure
        # canny_dists = [cv2.distanceTransform(cv2.bitwise_not(cv2.Canny(m, 0, 255)), cv2.DIST_L2, 3) for m in bgs_pp_masks_b]
        # canny_dists_preview = multi_preview(canny_dists)
        # cv2.imshow('canny_dists_preview', canny_dists_preview)
        # print(cannys[0].shape, cannys[0].dtype)

        if cv2.waitKey(int(1000/25)) != -1:
            break

    cv2.destroyAllWindows()
    for c in caps:
        c.release()
    if OUTPUT:
video_out.release()

\end{lstlisting}