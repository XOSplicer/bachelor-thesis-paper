% !TEX root = ../master.tex

\selectlanguage{ngerman}

\chapter*{Abstract}

\begingroup
  \begin{table}[h!]
    \setlength\tabcolsep{0pt}
    \begin{tabular}{p{3.5cm}p{10.0cm}}
      Titel & \dertitel \\
      Autor: & \derautor \\
    \end{tabular}
  \end{table}
\endgroup

Fahrstühle bewegen Menschen -- jeden Tag.
Die Fahrwege einer modernen Personenfahrstuhlanlage hängen ab von den Personen, die in jeder Etage auf ihre Beförderung zu unterschiedlichen Zielen warten sowie den Personen im Aufzug, die bereits auf dem Weg zu einer Zieletage sind. Aktuelle Systeme versuchen durch Angabe des Ziels beim Rufen des Aufzuges die Wege der Kabine für eine gleichverteilte und im Mittel schnelle Beförderung zu planen und Wartezeiten zu verringern sowie die Auslastung der Anlage zu optimieren.

In dieser Arbeit werden Möglichkeiten ergründet, diesen Prozess durch die Verwendung von Kamerasystemen in den Kabinen und vor den Einlässen zu unterstützen, indem durch Personen- und Objekterkennung die Anzahl der Passagiere und Wartenden sowie deren Platzbedarf und der Freiraum in der Kabine analysiert wird. 
Der Fokus liegt dabei auf der Analyse der benötigten und freie Volumina.
Dies umfasst eine Implementierung eines entsprechenden Bilderkennungssystems und eine modellhafte Darlegung eines Fahrstuhlsystems, das dieses verwendet.

TODO revise after finish


\selectlanguage{english}

\chapter*{Abstract}

\begingroup
  \begin{table}[h!]
    \setlength\tabcolsep{0pt}
    \begin{tabular}{p{3.5cm}p{10.0cm}}
      Title & \dertitel \\
      Author: & \derautor \\
    \end{tabular}
  \end{table}
\endgroup

\hspace{2cm}

Elevators move people -- every day.
The movement of an modern passenger elevator system depends on the persons on
each floor waiting to be transported to different destinations as well as the
passengers inside the elevator already on the way to their terminal location.
Recent systems try to optimize the movement of the elevator by utilizing
destination dispatch controls on each floor in such a way that travel
waiting times for passengers are minimized and the utilization of the elevator
is maximized.

This paper explores possibilities to support this process by the usage of
camera systems within the cabin and in front of the entrances which analyze
the amount and spatial needs of people waiting for the elevator as well as
currently occupying it.
The system employs techniques from the object and person recognition
from the field of computer vision.
The focus of the automated analysis is the spatial volume needed to
transport the passengers.
The work done for this paper includes an implementation of an appropriate
image recognition system and an exemplary outline of an elevator system
that uses such a system.

TODO revise after finish
