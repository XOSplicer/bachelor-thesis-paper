% !TEX root = ../master.tex

\selectlanguage{ngerman}

\chapter*{Abstract}

\begingroup
  \begin{table}[h!]
    \setlength\tabcolsep{0pt}
    \begin{tabular}{p{3.5cm}p{10.0cm}}
      Titel & \dertitel \\
      Autor: & \derautor \\
    \end{tabular}
  \end{table}
\endgroup

Fahrstühle bewegen Menschen -- jeden Tag.
Die Fahrwege einer modernen Personenfahrstuhlanlage hängen ab von den Personen, die in jeder Etage auf ihre Beförderung zu unterschiedlichen Zielen warten sowie den Personen im Aufzug, die bereits auf dem Weg zu einer Zieletage sind. Aktuelle Systeme versuchen durch Angabe des Ziels beim Rufen des Aufzuges die Wege der Kabine für eine gleichverteilte und im Mittel schnelle Beförderung zu planen und Wartezeiten zu verringern sowie die Auslastung der Anlage zu optimieren.

In dieser Arbeit werden Möglichkeiten ergründet, diesen Prozess durch die Verwendung von Kamerasystemen in den Kabinen und vor den Einlässen zu unterstützen, indem durch Personen- und Objekterkennung die Anzahl der Passagiere und Wartenden sowie deren Platzbedarf und der Freiraum in der Kabine analysiert wird. 
Der Fokus liegt dabei auf der Analyse der benötigten und freie Volumina.
Dies umfasst eine Implementierung eines entsprechenden Bilderkennungssystems und eine modellhafte Darlegung eines Fahrstuhlsystems, das dieses verwendet.

Das entwickelte Bilderkennungssystem verwendet mehrere Kameras 
und nutzt die Hintergrundsubtraktionstechnik um die Position von Personen und Objekten in der Kabine zu erfassen.
Ein shilhouettenbasierter Volumenschnitt wird durchgeführt um das eingenommene Volumen der Anwesenden zu erkennen.
Die so gewonnenen Daten können in einem angepassten Steuerungsalgorithmus für Aufzugsysteme genutzt werden, 
der großen Objekten in der Kabine Vorrang gewärt, um diese schneller zuzustellen.
Da diese Objekte die Kabine für andere Fahrgäste blockieren,
kann dies die Anzahl der beförderten Objekte erhöhen, 
was mit einer durchgeführten Simulation untermauert wird.
Eine solche Modifikation der Kotrollalgorithmus kann für Umfelder sinnvoll sein, in denen Passagiere und Objekte um Beförderung im Aufzug konkurrieren. 

\selectlanguage{english}

\chapter*{Abstract}

\begingroup
  \begin{table}[h!]
    \setlength\tabcolsep{0pt}
    \begin{tabular}{p{3.5cm}p{10.0cm}}
      Title & \dertitel \\
      Author: & \derautor \\
    \end{tabular}
  \end{table}
\endgroup

\hspace{2cm}

Elevators move people -- every day.
The movement of a modern passenger elevator system depends on the persons on
each floor waiting to be transported to different destinations as well as the
passengers inside the elevator already on the way to their terminal location.
Recent systems try to optimize the movement of the elevator by utilizing
destination dispatch controls on each floor in such a way that travel
waiting times for passengers are minimized and the utilization of the elevator
is maximized.

This paper explores possibilities to support this process by the usage of
camera systems within the cabin and in front of the entrances which analyze
the amount and spatial needs of people waiting for the elevator as well as
currently occupying it.
The system employs techniques from the object and person recognition
from the field of computer vision.
The focus of the automated analysis is the spatial volume needed to
transport the passengers.
The work done for this paper includes an implementation of an appropriate
image recognition system and an exemplary outline of an elevator system
that uses such a system.

The developed visual recognition system uses a multi-view camera set-up with the techniques of background subtraction to register the positions of passengers and objects in the elevator cabin.
The technique of silhouette-based volume intersection is used to detect the volume that these entities take up.
The data gathered by this system can be used for an elevator scheduling algorithm which gives priority to the delivery of large objects in the cabin, which blocks the space for other passengers.
A conducted simulation yields that this approach can improve the 
amount of delivered objects, which can be beneficial in an environment where passengers and cargo objects need to be moved likewise. 

