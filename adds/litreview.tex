% !TeX root = ../master.tex
\clearpage
\chapter*{Commented Bibliography}
\addcontentsline{toc}{chapter}{Commented Bibliography (Temporary)}

\newcommand{\parstartcite}[1]{\textbf{\citeauthor{#1}} \textsf{\citetitle{#1}} \autocite{#1} \newline}

\parstartcite{beyeler2017opencvml} is a an advanced, hands-on book about Machine Lerning with OpenCV.
OpenCV machine learning connects the fundamental theoretical principles
behind machine learning to their practical applications in a way that focuses
on asking and answering the right questions.
This book walks you through the key elements of OpenCV
and its powerful machine learning classes,
while demonstrating how to get to grips with a range of models that tackle real world problems.
\emph{Only relevant if ML is used.}


\parstartcite{crawford2017opencvpythonvideo} is a video course for OpenCV beginners that introduces basic usage of OpenCV 3 with Python 3. 
Learn about the data and image types unique to OpenCV, and find out how to
manipulate pixels and images. Instructor Patrick W. Crawford also shows how
to read video streams as inputs, and create custom real-time video interfaces.
Then comes the real power of OpenCV: object, facial, and feature detection.
\emph{Relevant for basic object detection}


\parstartcite{rusu2011pointcloud}
PCL presents an advanced
and extensive approach to the subject of 3D perception, and
it’s meant to provide support for all the common 3D building
blocks that applications need. The library contains state of the
art algorithms for: filtering, feature estimation, surface
reconstruction, registration, model fitting and segmentation.
\emph{Interesting when 3D reconstruction is needed.}

\parstartcite{levine1989microwave}
Early work on computer vision that uses a single camera setup to measure the volume of baked goods in real time. Used formula is $V = (\pi/4) \times A^2/h$, where area and height are constructed from pixel area and height with a calibration measurement.
\emph{Mildly relevant.}

\parstartcite{wang2017apple}
\emph{Mildly relevant.}

\parstartcite{hildebrand1997thickness}

\parstartcite{vogiatzis2010stereo}

\parstartcite{taj2010detection}

\parstartcite{hartley2004multiview}

\parstartcite{bostondiditeam2017mv3d}
\emph{Mildly relevant.}

\parstartcite{chen2017mv3d}
\emph{Mildly relevant.}

\parstartcite{saxena2007reconstruction}

\parstartcite{ge2010crowd}

\parstartcite{bayoa2009foreground}

\parstartcite{mordvintsev2013background}

\parstartcite{opencv2018histogram}
Usage of a histogram to normalize an image.
\emph{Possibly relevant for preprocessing of the video stream.}

\parstartcite{reddy2013foreground}

\parstartcite{zhang2017imageprocessing}

\parstartcite{grady2017graph}

\parstartcite{inoli2014imageprocessing}

\parstartcite{sun2010groupelevator}

\parstartcite{beers2015arrivals}
The paper performs a simulation of passenger flow and elevator control 
and deducts a method for an optimized waiting time when data about passenger 
arrivals is added to the system.
Besides the passenger waiting time, the riding time and the service time are measured. These
three together form the results from the passengers’ perspective. For the elevator’s
perspective the performance metrics are the number of stops, the maximum number of
passengers in the elevator and the energy consumption. The movement of the elevator is
simulated as well as the arrival of the passengers. A set of rules about the movement of the
elevator is called a policy. In this paper 2 different policies with different settings are evaluated
for both up-peak and down-peak traffic. Several realistic scenarios are simulated. In all
scenarios the building consist 11 floors and one elevator with unlimited capacity.
\emph{Very relevant for method of simulation, consideration of addidional information, and nice state of the art chapter.}

\parstartcite{kwon2014sensor}
Usage of smart home sensors to detect behavior patterns in a apartment in order to predict the call and usage of an elevator. Improved waiting time by issuing a reservation call earlier.
\emph{Medium relevant for its up-to-dateness.}

\parstartcite{zhou2018monitoring}
Introduces a high-level overview how IoT can be used to monitor a complete elevator system.
\emph{Mildly relevant.}


\parstartcite{siikonen1997models}

\parstartcite{unger2015aufzuege}
Extensive book with a high level overview regarding elevators and which technical solutions are employed to build and run them. A broad classification regarding the implementation details is given such as movement, sensors, interaction elements and control units are given. Major elevator companies are listed. \emph{Relevant for basics in elevator systems.}

\parstartcite{li2004background}

\parstartcite{piccardi2004background}

\parstartcite{huwer2000change}

\parstartcite{kaewtrakulpong2001background}

\parstartcite{sonaten2011volume}
Stackoverflow answer that illustrates multiview geometry Reconstruction with 2D cameras.
\emph{Relevant for illustration purposes.}

\parstartcite{siikonen1993simulation}


\parstartcite{hakonen2008simulation}

\parstartcite{hakonen2004optimization}

\parstartcite{siikonen2000simulation}

\parstartcite{susi2004passenger}

\parstartcite{horejsi2011dispatch}

\parstartcite{sapkal2017volume}
Proposes an method for volume estimation without the need for 3D reconstruction.
The method needs 2--3 images and incorporates edge
detection, image segmentation, and feature extraction to
identify the object and find its dimensions, after which,
the object is broken down into infinitesimally slices along
the horizontal axis and volume is calculated for each
slice. The addition of all volumes of these slices results in
the estimated volume of the object.
An calibration object is needed.
The method assumes a round object and each horizontal slice area is approximated by 
$V_{slice} = \pi \times \frac{a}{2} \times \frac{b}{2} \times \delta{}h$ 
where $a$ and $b$ are the diameters of the object from  two sides and $\delta{}h$ is the height of the slice.
\emph{Relevant for simple volume estimation.}

\parstartcite{lin2011control}

\parstartcite{zhang2016door}

\parstartcite{xang2016trafficlist}

\parstartcite{sorsa2005destination}

\parstartcite{axelsson2013strategies}
Bachelor essay that describes elevator control strategies including
Collective control, zone approach, search-based non-greedy, rule-based and genetic algorithms. A collective control and zone approach algorithm is compared by running a discrete elevator simulation for up and down peak scenarios based on a fixed model of a building. \emph{Relevant for simulational approach and overview over trategies.}

\parstartcite{suzuki1994number}

\parstartcite{effati2015strategies}
Supports the findings of \textcite{axelsson2013strategies} by performing a similar comparison and simulation, but also includes random traffic patterns. 

\parstartcite{khomytskyi2016simulation}

\parstartcite{omar2013simulation}

