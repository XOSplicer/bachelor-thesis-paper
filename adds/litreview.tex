% !TeX root = ../master.tex
\clearpage
\chapter*{Commented Bibliography}
\addcontentsline{toc}{chapter}{Commented Bibliography (Temporary)}

\newcommand{\parstartcite}[1]{\textbf{\citeauthor{#1}} \textsf{\citetitle{#1}} \newline}

\parstartcite{beyeler2017opencvml} is a an advanced, hands-on book about Machine Lerning with OpenCV.
OpenCV machine learning connects the fundamental theoretical principles
behind machine learning to their practical applications in a way that focuses
on asking and answering the right questions.
This book walks you through the key elements of OpenCV
and its powerful machine learning classes,
while demonstrating how to get to grips with a range of models that tackle real world problems.


\parstartcite{crawford2017opencvpythonvideo} is a video course for OpenCV beginners that introduces basic usage of OpenCV 3 with Python 3. 
Learn about the data and image types unique to OpenCV, and find out how to
manipulate pixels and images. Instructor Patrick W. Crawford also shows how
to read video streams as inputs, and create custom real-time video interfaces.
Then comes the real power of OpenCV: object, facial, and feature detection.
Learn how to leverage the image-processing power of OpenCV using methods like
template matching and machine learning data to identify and recognize features.


\parstartcite{rusu2011pointcloud}

\parstartcite{levine1989microwave}
Early work on computer vision that uses a single camera setup to measure the volume of baked goods in real time. Used formula is $V = (\pi/4) \times A^2/h$, where area and height are constructed from pixel area and height with a calibration measurement.
\emph{Mildly relevant}

\parstartcite{wang2017apple}

\parstartcite{hildebrand1997thickness}

\parstartcite{vogiatzis2010stereo}

\parstartcite{taj2010detection}

\parstartcite{hartley2004multiview}

\parstartcite{bostondiditeam2017mv3d}

\parstartcite{chen2017mv3d}

\parstartcite{saxena2007reconstruction}

\parstartcite{ge2010crowd}

\parstartcite{bayoa2009foreground}

\parstartcite{mordvintsev2013background}

\parstartcite{opencv2018histogram}

\parstartcite{reddy2013foreground}

\parstartcite{zhang2017imageprocessing}

\parstartcite{grady2017graph}

\parstartcite{inoli2014imageprocessing}

\parstartcite{sun2010groupelevator}

\parstartcite{beers2015arrivals}

\parstartcite{kwon2014sensor}

\parstartcite{zhou2018monitoring}

\parstartcite{siikonen1997models}

\parstartcite{unger2015aufzuege}
Extensive book with a high level overview regarding elevators and which technical solutions are employed to build and run them. A broad classification regarding the implementation details is given such as movement, sensors, interaction elements and control units are given. Major elevator companies are listed. \emph{Relevant for basics in elevator systems.}

\parstartcite{li2004background}

\parstartcite{piccardi2004background}

\parstartcite{huwer2000change}

\parstartcite{kaewtrakulpong2001background}

\parstartcite{sonaten2011volume}
Stackoverflow answer that illustrates multiview geometry Reconstruction with 2D cameras.

\parstartcite{siikonen1993simulation}

\parstartcite{hakonen2008simulation}

\parstartcite{hakonen2004optimization}

\parstartcite{siikonen2000simulation}

\parstartcite{susi2004passenger}

\parstartcite{horejsi2011dispatch}

\parstartcite{sapkal2017volume}

\parstartcite{lin2011control}

\parstartcite{zhang2016door}

\parstartcite{xang2016trafficlist}

\parstartcite{sorsa2005destination}

\parstartcite{axelsson2013strategies}
Bachelor essay that describes elevator control strategies including
Collective control, zone approach, search-based non-greedy, rule-based and genetic algorithms. A collective control and zone approach algorithm is compared by running a discrete elevator simulation for up and down peak scenarios based on a fixed model of a building. \emph{Relevant for simulational approch.}

\parstartcite{suzuki1994number}

\parstartcite{effati2015strategies}

\parstartcite{khomytskyi2016simulation}

\parstartcite{omar2013simulation}

