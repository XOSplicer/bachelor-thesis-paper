% !TeX root = ../master.tex
\clearpage
\chapter*{Commented Bibliography}
\addcontentsline{toc}{chapter}{Commented Bibliography}

\newcommand{\parstartcite}[1]{\textbf{\citeauthor{#1}} \textsf{\citetitle{#1}} \newline}

\parstartcite{beyeler2017opencvml} is a an advanced, hands on book.
OpenCV machine learning connects the fundamental theoretical principles
behind machine learning to their practical applications in a way that focuses
on asking and answering the right questions.
This book walks you through the key elements of OpenCV
and its powerful machine learning classes,
while demonstrating how to get to grips with a range of models.
This book targets Python programmers who are already familiar with OpenCV;
this book will give you the tools and understanding required to build your
own machine learning systems, tailored to practical real-world tasks

\parstartcite{crawford2017opencvpythonvideo} is a video course for OpenCV beginners that introduces basic usage of OpenCV with python.
OpenCV is an open-source toolkit for advanced computer vision.
It is one of the most popular tools for facial recognition,
used in a wide variety of security, marketing, and photography applications,
and it powers a lot of cutting-edge tech, including augmented reality
and robotics. This course offers Python developers a detailed introduction to
OpenCV 3, starting with installing and configuring your Mac, Windows,
or Linux development environment along with Python 3.
Learn about the data and image types unique to OpenCV, and find out how to
manipulate pixels and images. Instructor Patrick W. Crawford also shows how
to read video streams as inputs, and create custom real-time video interfaces.
Then comes the real power of OpenCV: object, facial, and feature detection.
Learn how to leverage the image-processing power of OpenCV using methods like
template matching and machine learning data to identify and recognize features.

\parstartcite{rusu2011pointcloud}

\parstartcite{levine1989microwave}
Early work on computer vision that uses a single camera setup to measure the volume of baked goods in real time. Used formula is $V = (\pi/4) \times A^2/h$, where area and height are constructed from pixel area and height with a calibration measurement.

\parstartcite{wang2017apple}

\parstartcite{hildebrand1997thickness}

\parstartcite{vogiatzis2010stereo}

\parstartcite{taj2010detection}

\parstartcite{hartley2004multiview}

\parstartcite{bostondiditeam2017mv3d}

\parstartcite{chen2017mv3d}

\parstartcite{saxena2007reconstruction}

\parstartcite{ge2010crowd}

\parstartcite{bayoa2009foreground}

\parstartcite{mordvintsev2013background}

\parstartcite{opencv2018histogram}

\parstartcite{reddy2013foreground}

\parstartcite{zhang2017imageprocessing}

\parstartcite{grady2017graph}

\parstartcite{inoli2014imageprocessing}

\parstartcite{sun2010groupelevator}

\parstartcite{beers2015arrivals}

\parstartcite{kwon2014sensor}

\parstartcite{zhou2018monitoring}
