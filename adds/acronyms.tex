% !TEX root = ../master.tex
\clearpage
\chapter*{List of Acronyms}
\addcontentsline{toc}{chapter}{List of Acronyms}

%Verwendung:
%		\ac{Abk.}   --> fügt die Abkürzung ein, beim ersten Aufruf wird zusätzlich automatisch die ausgeschriebene Version davor eingefügt bzw. in einer Fußnote (hierfür muss in header.tex \usepackage[printonlyused,footnote]{acronym} stehen) dargestellt
%		\acs{Abk.}   -->  fügt die Abkürzung ein
%		\acf{Abk.}   --> fügt die Abkürzung UND die Erklärung ein
%		\acl{Abk.}   --> fügt nur die Erklärung ein
%		\acp{Abk.}  --> gibt Plural aus (angefügtes 's'); das zusätzliche 'p' funktioniert auch bei obigen Befehlen
%	siehe auch: http://golatex.de/wiki/%5Cacronym

\begin{acronym}[MMMM]
    \acro{2D}{2 dimensional}
    \acro{3D}{3 dimensional}
    \acro{4D}{4 dimensional}
	\acro{API}{application programming interface}
	\acro{BDSG}{Bundesdatenschutzgesetz}
    \acro{BSD}{Berkeley Software Distribution}
	\acro{CPU}{central processing unit}
	\acro{CV}{Computer Vision}
	\acro{DHBW}{Baden-Wuerttemberg Cooperative State University}
	\acro{EU}{European Union}
	\acro{GDPR}{General Data Protection Regulation}
	\acro{IP}{Internet Protocol}
	\acro{IT}{Information Technology}
	\acro{MOG}{Mixture of Gaussians}
	\acro{RAM}{Random Access Memory}
    \acro{RGB}{red-green-blue}
    \acro{SI}{International System of Units}
    \acro{UML}{Unified Modeling Language}
    \acro{YAML}{YAML Ain't Markup Language}
\end{acronym}
